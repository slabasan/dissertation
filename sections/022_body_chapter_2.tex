%--- Chapter 2 ----------------------------------------------------------------%
\chapter{Power and Performance Tradeoffs Under Reduced Clock Frequencies}
\label{ch:ldav}

Most of the text in this chapter comes from \cite{7348074}, which was a
collaboration between Matthew Larsen (LLNL), Hank Childs (UO), and myself.
%
The writing of this paper was a collaboration between Hank Childs and myself,
and I performed the lead role on all writing, especially including the
description of experiments and results.
%
Matthew Larsen and I wrote the benchmark tests.
%
Hank Childs and Matthew Larsen provided the data sets in different sizes and
layouts.
%
Hank Childs and I developed the algorithm implementations.
%
I developed the performance monitoring infrastructure, designed and executed
the study, and contributed to the majority of the other sections.

This chapter explores the power and performance tradeoffs for one common
visualization routines.
%
In this chapter, we focus specifically on isosurfacing, a canonical
visualization algorithm, where the output is a three-dimensional surface
containing points of a constant value.
%
We ran an extensive study to understand changes in execution behaviors when the
CPU clock frequency is reduced.
%
Findings from this study begin to inform how we can tune algorithmic-level knobs
to optimize energy and power usage.

\section{Motivation}
\input{sections/chapter2-ldav15/introduction.inc}

\section{Related Work}
\label{sec:ch2-relwork}
\input{sections/chapter2-ldav15/relwork.inc}

\section{Benchmark Tests}
\label{sec:ch2-benchmarks}
\input{sections/chapter2-ldav15/background.inc}

\section{Experimental Setup}
\input{sections/chapter2-ldav15/study.inc}

\section{Results}
\input{sections/chapter2-ldav15/results_intro.inc}

\section{Summary}
\input{sections/chapter2-ldav15/conclusion.inc}
